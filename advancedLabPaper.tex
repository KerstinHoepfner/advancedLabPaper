\documentclass[12pt]{article}
\usepackage{hyperref}
\usepackage{graphicx}
\usepackage{indentfirst}
\usepackage{setspace}
\usepackage{xcolor}
\usepackage[export]{adjustbox}
\usepackage{subfig}
\usepackage{authblk}
\setlength{\parskip}{1em}


\topmargin=0.0in
\oddsidemargin=0.0in
\evensidemargin=0in
\textwidth=6.5in
\marginparwidth=0.5in
\headheight=0pt
\headsep=0pt
\textheight=9.0in

\title{\textbf{Modern Techniques to Engage Advanced Laboratory Students using MicroController Hardware and Python}}

\author[1]{Shawn Zaleski}
%%\author[1]{Thomas Hebbeker}
\author[1]{Kerstin Hoepfner}
%%\author[1]{Henning Keller}
%%\author[1]{Giovanni Mocellin}
\affil[1]{III Physikalisches Institut A RWTH Aachen University}
%%\institute{RWTH-Aachen}
%%\institute{III Physikalisches Institut A RWTH Aachen University}
\date{\empty}
\pagestyle{empty}

\begin{document}
\maketitle
%%\centerline{Gas Electron Chambers in University Advanced Physics Laboratory}
\noindent
%%\line(1,0){470}
\newline

\begin{abstract}
Engaging students in a physics lab setting can be challenging, especially while modernizing the experiments currently available. Often, students complain about out-dated equipment or ``cookie cutter" type experiments that do not allow much creative freedom on the part of the student. The advanced undergraduate lab course experiments for particle physics at RWTH Aachen University were updated using inexpensive DIY components such as Arduino and Raspberry Pi. To engage students better using Arduino and Python, coding can be left completely to the students to write or they may be supplied with partial, or complete, working programs. This allows the instructor to tailor the lab to the appropriate skill level of the students. We outline two different implementations in lab experiments. One where the Raspberry Pi and Arduino collect Geiger-Mueller data, and another that allows them to collect data with the Arduino via Bluetooth on their smartphone. Student feedback will be presented.

\end{abstract}

\newpage

\section{Introduction}
The Bachelor's advanced physics laboratory at RWTH Aachen University is offered annually to upper division physics students, typically in their final year of physics study.
This laboratory serves roughly 150 students during each offering.
The laboratory is structured largely into two halves.
During the first half of the semester students attend a series of 12 lectures on high energy physics (HEP) detector topics and 12 lectures on solid state physics (SSP)detector topics.
During the second half of the semester students, in pairs of groups of 2 or three students each, perform 5-7 experiments.
These expeiriments are selected from a set of approximately 25 available experiments of which roughly half are from HEP topics and the other half from SSP topics.
The experiments that the students perform range from classical large-scale experiments such as the Stern-Gerlach experiment to more complex and modern topics such as the gas electron multiplier experiment.


The structure of the experimental portion of each laboratory experiment is broken into three primary parts: discussion, experiment, and report.
Students are expected to have attended the approrpiate lecture during the first half of the class as well as to have read the experiment manual prior to performing the experiment.
As such, students are to respond verbally to questions pertaining to the physics theory and electronics hardware operation outlined in the manual and lecture.
An experiment supervisor asks the students the discussion questions in a free response format.
If the students do not demonstrate the minimum level of understanding required to safely perform the experiment, the students are asked to return after reviewing the material further, at which point they are questioned again.
The experiment supervisor then explains, and if necessary demonstrates, any part of the experiment that might be easily damaged, or require special care not described in the manual.
From this point, the students are left to perform the experiment and collect data as outlined in the lab manual.
If the students have questions, encounter problems, or otherwise need to speak to the supervisor, they meet with him as needed.
Lastly, within two weeks of completing the experiment, the students are expected to submit a formal lab report which should take a form similar to that of a typical physics paper.

This paper details some of improvements that were implemented into the advanced physics laboratory course.
In section 2 we motivate these improvements.
In setcions 3, 4, and 5, we give detailed descriptions of the portion of experiements that were modified, discussing why the portions were modified and what changes were implemented.
In section 6 we discuss some conclusions found and general student feedback collected following implementation.

\section{Motivation}
To address student complaints within the advanced lab...

\section{Gas Detector and Statistics}
An update to the Gas Detectors and Statistics Laboratory....

\section{Detector Principles}
An update to the Multiple Scattering portion of the Detector Principles experiment...

\section{Stern-Gerlach Experiment}
An update to the readout of the Stern-Gerlach Experiment...

\section{Conclusions}
After making these changes, we found the following from the students...

\section{Summary}
To sum up...



\end{document}
