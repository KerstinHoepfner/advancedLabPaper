\section{Gas Detector and Statistics}
The Statistics portion of the Gas Detector and Statistics experiment was the first experiment in which the new microcontroller technology was implemented.
In this experiment students measure the number of counts registered by a GM counter using a fixed time window of the order of 1-10 $s$ depending on the strength of the source used.
The number of decays from a Sr-90 source and the number of incident cosmic ray muons registered as counts in the time window are used to generate histograms.
The histograms form a well defined (for sufficient number of collections) Gaussian distribution in the case of beta decays, and Poisson distribution in the case of the cosmic rays.
The students then perform a statistical analysis consisting of finding the mean and standard deviation using standard statistical procdures as well as invoking a $\chi^2$-based analysis.

The primary reason for updating this setup is that the traditional gas GM that was used no longer functioned properly and was expensive to replace.
Another reason to move to an updated data acquisition setup was that the NIM crate and supporting modules used to power the GM and register counts required students to record the data using paper and pencil.
Students have complained that this mode of data acquistion is out-dated, and it is true that many modern experiments use some form of computer, or other electronics, based data acquistion and logging.


